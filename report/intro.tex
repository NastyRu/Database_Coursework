\Introduction

\hfill

	На сегодняшний день более 4,5 миллиарда людей пользуются интернетом, а аудитория социальных сетей перевалила за отметку в 3,8 миллиарда \cite{useinternet}.
	
	Получается, что общение посредством мессенджеров является неотъемлемой частью жизни современного человека. 
	
	Практически все компании сталкиваются с вопросом: <<Какой использовать корпоративный мессенджер?>>. Он должен объединять все внутренние и внешние коммуникации в одно пространство. Сегодня рынок корпоративных мессенджеров предлагает десятки вариантов с разным функционалом, интеграцией с другими сервисами и ценовой политикой. Такие как, Slack, Microsoft Teams, Тwist, Discord и другие \cite{corporatemessengers}. 
	
	Однако данные приложения имеют ряд недостатков:
	\begin{itemize}
	\item сотрудники могут пользоваться различными мессенджерами;
	\item в контактах могут присутствовать люди, не имеющие отношения к работе, что служит отвлекающим фактором;
	\item недостаточная функциональность;
	\item безопасность. 
	\end{itemize}
	
	Для удобства коммуникации и работы внутри компании разрабатываются свои корпоративные внутренние средства общения, которые компенсируют вышеописанные недостатки. 
		
	Таким образом можно сделать вывод, что создание универсального мессенджера является актуальной темой, так как данное приложение будет ориентировано на специфику конкретной компании.
	
	Целью данной работы является реализация микросервиса универсального мессенджера.  Функциональное назначение микросервиса --- хранение и работа с данными.
	
	Для достижения поставленной цели необходимо решить следующие задачи. 
	\begin{enumerate}
		\item[1. ] Анализ существующих решений. 
		\item[2. ] Проектирование микросервиса мессенджера. 
		\item[3. ] Реализация микросервиса. 
		\item[4. ] Разработка тестового клиентского приложения. 
	\end{enumerate}



